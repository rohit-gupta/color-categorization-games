%% LyX 2.0.2 created this file.  For more info, see http://www.lyx.org/.
%% Do not edit unless you really know what you are doing.
\documentclass[11pt,a4paper,english]{article}
\usepackage[cp1250]{inputenc}
\pagestyle{plain}

\makeatletter

%%%%%%%%%%%%%%%%%%%%%%%%%%%%%% LyX specific LaTeX commands.
\pdfpageheight\paperheight
\pdfpagewidth\paperwidth


%%%%%%%%%%%%%%%%%%%%%%%%%%%%%% User specified LaTeX commands.

\usepackage[english]{babel}
\usepackage{a4wide}

\makeatother

\usepackage{babel}
\begin{document}

\title{Modeling basic color term usage\\
as similarity-maximization games\\
{\large Final project for ``Language and Games''}}


\author{Jos� Pedro Correia\\
(No. 10232095) MSc Logic \and Radek Ocel�k\\
(No. XXXXXXX) MSc Logic}

\maketitle

\section{The World Color Survey and the facts to be explained}

Berlin and Kay \cite{Berlin69} pioneered a long tradition of research
into color term systems of world's languages. This broad typological
study pointed out that there clearly are universal tendencies concerning
repertoires of basic color terms in particular languages. The essential
claim for our purposes is an evolutionary one: there is a cross-linguistically
universal order in which color vocabularies are enriched by new terms.
The order of emergence is captured by the following implicational
hierarchy of universals:

{[}white, black{]} $<$ {[}red{]} $<$ {[}green, yellow{]} $<$ {[}blue{]}
$<$ {[}brown{]} $<$ {[}purple, pink, orange, gray{]}

This hierarchy expresses a claimed complex constraint on all existing
natural languages. It should be read as follows: if a language has
a well-established term for red (that is, a term covering the point
of the color space which is the prototypical denotation of the English
�red�), it also has basic terms for white and for black, but not necessarily
\textit{vice versa}. Similarly, if it has one for green or for yellow,
it also has one for red, and so on. According to this picture, color
term systems are far from being relativistically arbitrary: whatever
number of terms they contain, these terms can carve the color space
only in certain ways. For example, in a language with three color
terms, these are bound to cover, respectively, red, black, and white.
Full partition of the color space is claimed, so the English terms
are in fact misleading. In the last case, the term for red is likely
to cover violet or orange as well and the other two terms would rather
correspond to �dark� and �light�.

In order to examine the main theses of the founding study, the World
Color Survey (WCS, a comprehensive monograph is \cite{WCS}) was conducted
in the subsequent decades. It substantially broadened the empirical
base and improved the methodology of the previous work, performing
field research for 110 unwritten languages (listed in \cite{Reg05})
with a negligible level of genetic interrelatedness, with 24 informants
per language on average. (Cf. \cite{KayEnc} for methodology.) The
employed color system was one of 330 Munsell chips, 320 of them in
the Lenneberg and Roberts array of 40 hue columns and 8 levels of
lightness, at maximum saturation, plus an achromatic column of 10
chips from white to black.

The results of the WCS concerning the universality of color terms
emergence can be found in Kay and Maffi \cite{Kay99}. Here, the empirically
documented color term systems are classified into 9 types in 5 stages
with respect to how six focal points of the color space (prototypical
denotations of the English �white�, �red�, �yellow�, �green�, �blue�,
�black�) are grouped by the vocabulary of each particular language.
For instance, in Stage II (languages with 3 basic color terms) there
is only one observed type, {[}white; red+yellow; black+green+blue{]};
in Stage III (4 color terms) there are the types {[}W; R; Y; Bk+G+Bu{]},
{[}W, R+Y, G+Bu, Bk{]} and {[}W, R, Y+G+Bu; Bk{]}. The authors note
that there are five possible evolutionary trajectories between stages
I to V, assuming that any evolutionary step consists in splitting
the denotation (the covered focal points) of one term of the previous
system in two. The trajectory {[}W+R+Y; Bk+G+Bu{]} $>$ {[}W, R+Y,
Bk+G+Bu{]} $>$ {[}W, R+Y, G+Bu, Bk{]} $>$ {[}W; R; Y; G+Bu; Bk{]}
$>$ {[}W; R; Y; G; Bu; Bk{]} is the most represented one in the WCS,
capturing 91 of the 110 languages.%
\footnote{As one type can be shared by more trajectories, the languages captured
by all these trajectories do not add up to 110, or 100\%.%
}

The way the empirical results are presented requires some discussion.
The talk of evolutionary paths as instantiated by the languages is
slightly misleading: what was really observed was in each case a static
color term system belonging to one of the 9 types, or to a transition
between two of them. These observed transitions are the maximum of
diachrony captured in the WCS; apart from this we cannot infer anything
about how a particular observed color vocabulary actually came about.
This having been clarifed, the formulation in terms of these emergence
trajectories and the representation of the particular types seems
more appropriate than the original strong formulation in terms of
implicational hierarchy. First, it draws attention to the almost universal%
\footnote{Rare exceptions are discussed in section 3 of \cite{Kay99}.%
} principle of partitioning the color space by the available terms,
and consequently to the fact that if a color vocabulary is enriched
with an additional term, the denotation of some or all of the already
established terms is likely to be modified. Second, the gathered data
do not seem univocal enough to formulate an implicational hierarchy
as strong as that of Berlin and Kay \cite{Berlin69}, and anyway,
in order to formulate any such generalization, the data would have
to be statistically evaluated with this in mind. Admittedly, an overall
quantitative evaluation of the WCS data has been done \cite{Kay03,Reg05}
and it has shown a clear non-random match among color term systems
of world's languages, thus refuting the position of full relativism
in this respect. But whether \textit{particular} generalizations (implicative
or other) are valid is an altogether different question. We conjecture
that a consequential part of the original hierarchy would not find
a significant support in the data, since in the sample there are relatively
little languages spread among the types with 4 or less color terms,
as opposed to about 80 languages with 5 terms, 6 terms or inbetween.

In the general lack of statistically conclusive support for individual
universal features of human color categorization, we will focus on
the two of them that can be most reliably inferred from the absolute
numbers reported in Kay and Maffi \cite{Kay99}. One is that 3-term
vocabularies tend to partition the color space according to the scheme
{[}W; R+Y; Bk+G+Bu{]}, that is, to separate the warm colors while
keeping the cool colors together with black. This is the only reported
type for Stage II, instantiated by 6 languages. The other universal
feature to focus on is the evolutionarily late division of green and
blue: {[}W; R; Y, G+Bu; Bk{]} is by far the most represented type
of 5-term systems (Stage IV), instantiated by 41 languages. The strong
universal tendencies to carve up the color space in the described
way when, respectively, three and five color terms are available will
be in the following regarded as the most obvious, or secure, empirical
facts to be explained. Besides qualitative evaluation based on this
finding, we will evaluate our model in a more refined, quantitative
way, against the detailed WCS data for particular languages \cite{WCSdata}.%It could be even argued that a model which can account for these specific facts will, in turn, be able to indicate something about the evolutionary stages, such as the hypothetical Stage I, for which empirical evidence is missing.



\section{Related work}

The general debate on the nature of cognitive categories is dominated
by three competing paradigms, nativism, empiricism and culturalism,
the third often presented as a solution to the aged dilemma between
the first two (cf. \cite{Steels05} and the references there). The
debate on color categorization, specifically, is furthermore structured
along the dimension of universality vs. relativity, which is arguably
a distinct one, despite affinities such as that between nativism and
universalism. The WCS has posed this question as straightforwardly
empirical and provided data; as a result, recent positions on both
the universalist \cite{Kay03,Reg05} and the relativist \cite{Rob05}
side are rather moderate.

Granted that there \textit{are} universal tendencies in color categorization,
explanation of these (and of the remaining relativity) has been approached
in several ways. Kay and Maffi \cite{Kay99} themselves present an
updated version of a model that had been continuously developed by
the WCS authors, on the (close to) nativist assumption of 6 naturally
focal colors. The issue has been also studied within the broadly culturalist
framework of the Iterated Learning Model \cite{Smith03,Dowm07}. However,
here we will only discuss in detail the approaches that directly motivate
our own model, in which the emergence of categories is conceived in
terms of cultural interaction on the basis of innate characteristics
of human perception. In a nutshell, these are works \cite{Baron10}
and \cite{Lor12}, focusing on the impact of perceptual constraints
on routinized cultural interaction; the more recent work \cite{Reg07}
of the WCS authors investigating partitions of the perceptual space
in terms of optimality; and J�ger and van Rooij's \cite{Rooij07}
proposal to treat the issue in game-theoretical terms. Moreover, in
the final section on prospects we discuss one additional, empiricist
approach, to motivate a possible extension of our model.

The first two of our motivating approaches jointly assume that universality
of categorization might be explainable in terms of specific characteristics
of human visual perception. We discuss, first, Baronchelli et al.
\cite{Baron10} and Loreto et al. \cite{Lor12}, who attempt to derive
the universals from a particular formulation of the dependence of
perception on the physical character of the input. Then we outline
the explanatory strategy of Regier, Kay and Khetarpal \cite{Reg07},
which appears to be a more general, though in a sense less elaborated,
version of the former.


\subsection{Just Noticeable Difference}

\cite{Baron10}, as well as \cite{Lor12}, appeal in explanation to
a simple characteristic of human visual perception, called Just Noticeable
Difference (JND). The human JND is a psychophysiologically determined
function which for any given wavelength from the extent of the visible
spectrum gives the minimal difference in wavelength of two hues that
are distinguishable by human eye in that particular region of the
spectrum. This function is implemented as a constraint on cultural
interaction of artificial agents, conceived roughly along the lines
of Steels and Belpaeme's \cite{Steels05} ''culturalist model''.
In this setting, color term vocabularies and categorical systems of
individual agents in a population are made to co-evolve through their
repetitive participation in standardized linguistic interaction over
empirical input (''the Category Game''). Similarity between the
emergent systems and those observed empirically is then supposed to
vindicate the explanatory role of the human JND.

Despite general formulations (�excellent quantitative agreement� with
the WCS data), Baronchelli et al. are successful only in a specific
sense. Their simulation does not demonstrate for particular universal
features of human color categorization how these might have been arrived
at. It shows only that categorical systems developed via cultural
interaction constrained by the human JND are less dispersed across
populations than when a flat, non-human JND is used. The only quantitative
agreement, then, is between the ratio of the two respective values
of dispersion, and the dispersion ratio of the actually observed categorical
systems of the WCS, compared to a specific randomization of these
(as in \cite{Kay03}). The agreement of these two ratios on $\sim$1.14
is remarkable, but hard to interpret in isolation. %both these values are construed in such a complicated fashion that proper interpretation of the match can hardly be straightforward. RO: I insist upon this, though it may not sound as a correct argument. The thing is that evolution in silica, according to a specific dynamic, is something quite different and trasparent in comparison with the overall evolution of world's languages with the immense number of participating factors (including genetic relations, language contact, wars and whatever). While in some respects the artificial dynamic may be almost reasonable model for the latter, I'm pretty sure that from an out-of-the-blue coincidence of two similarly constructed numbers describing outcomes of these two processes absolutely nothing can be inferred.


Loreto et al. \cite{Lor12} come somewhat closer to explanation of
particular universals of color categorization. The human JND as a
constraint on routine language interaction over empirical input is
sufficient for them to derive a hierarchy of color terms according
to the time it takes for color terms in various regions of the visible
spectrum to be agreed upon within the population. The announced �excellent
quantitative agreement with the empirical observations of the WCS�
is concealed from the reader. But the authors rightly point out that
their hierarchy, {[}red, (magenta)-red{]} $<$ {[}violet{]} $<$ {[}green/yellow{]}
$<$ {[}blue{]} $<$ {[}orange{]} $<$ {[}cyan{]}, is similar to the
implicational hierarchy of Berlin and Kay \cite{Berlin69}. Let us
discuss the relevance of this finding.

First, there seems to be a methodological problem with choice of color
terms and their matching to regions of the spectrum. This should,
arguably, have been done either by selecting a set of cross-linguistically
basic colors and locating them in the spectrum, or by selecting important
points or sections of the JND function and reading off the respective
colors; but an opaque combination of both seems to have taken place.
In the first case we would expect both green and yellow in the selection,
instead of green/yellow, and we might challenge the inclusion of cyan
and (magenta)-red. In the second case, while most of the selected
points reflect peaks and valleys of the function, (magenta)-red does
not, violet and red are disputable, and there is an unreflected valley
between violet and blue. Some of this could be actually resolved in
favor of the parallel between the achieved hierarchy and Berlin and
Kay's hierarchy; first of all, there are reasons to pick only red
for the experiment, instead of red, (magenta)-red and violet. But
there remains the problem that green/yellow in the achieved hierarchy
is a single transitional color, while in Berlin and Kay's hierarchy
green and yellow are two distinct colors occupying the same position.

Moreover, let us remind that Berlin and Kay \cite{Berlin69} is a
dated reference and there is little point in evaluating explanatory
proposals concerning universals of color categorization against the
hierarchy stated there, in presence of the WCS data, the superiority
of which is both empirical and methodological. Our conclusions in
Section ... indicate that the mismatch between Loreto, Mukherjee and
Tria's \cite{Lor12} actual findings and the cross-linguistic reality
would have been magnified by an up-to-date evaluation, rather than
attenuated. While we believe that the features of human perception
that are captured by the JND function should play an important explanatory
role regarding linguistic universals, the two papers just discussed
do not more than indicate so.


\subsection{The CIELAB space}

The previous approach appeals to a particular feature of human perception
(the resolution power in different frequencies of visible light).
The explanatory strategy adopted by Regier, Kay and Khetarpal \cite{Reg07},
with reference to \cite{Jam97}, is a more general version of that.
Instead of carving up a physically defined space (one-dimensional
in the previous case), they consider partitions of the psychologically
relevant, 3D color space CIELAB, which is designed so that standard
Euclidean distance of two hues corresponds to their psychological
dissimilarity. In this, the human JND is encompassed rather than discarded
as a source of explanation, for what it expresses has to be involved
also in construction of any psychologically relevant space. When the
Munsell color palette used in the WCS is projected into the CIELAB
space, its chips mark the surface of an irregular sphere there. What
is then discussed are partitions of the set of color points thus arranged.
The authors convincingly show a strong preference of the WCS languages
for efficient partitions, efficient in terms of maximizing the compactness
of their color categories in the CIELAB space. What this means is
that the closer two chips are in the perceptual space, the more likely
they will be lumped under the same color term.

This is clearly an important result, pointing to optimality as an
essential factor of color categorization. However, this line can be
drawn further. Given the specific way of evaluation (each language's
actual partition vs. its various rotations around the sphere), the
results cannot directly account for any particular linguistic universal
in question. For instance, we do not see whether the most efficient
ways of partitioning the figure into 5 regions involve keeping blue
and green together. Another issue is that optimality or efficiency
is a static feature of a categorical system (of an individual speaker
or of a language \textit{in abstracto}), without it being clear how
it might have come about. In our approach we adopt the idea of the
overall character of human visual perception, reflected in the CIELAB
color space, as the likely source of universals of color categorization.
For sake of comparability with the WCS data we also work with the
projected Munsell palette. Instead of static assessment of optimality,
though, we will be interested in an evolutionary, agent-based dynamic
of cultural interaction, in the game-theoretic formulation proposed
in \cite{Rooij07}, over empirical input located in the defined space.


\subsection{Similarity-maximization games}

J�ger and van Rooij \cite{Rooij07} construe the problem as a similarity-maximization
signaling game. Nature picks a point from the color space as the meaning
to be conveyed; the sender sends one term from a finite set to signal
the chosen meaning to the receiver; the receiver interprets the received
signal by choosing a point from the color space. The payoff this signalling
action brings to both the sender and the receiver is a monotonically
decreasing function of the distance of the receiver's interpretation
from the intended meaning in the color space. In general, the sender
strategy is a function from the set of points of the color space to
(a probabilistic distribution over) the given set of terms, and the
receiver strategy is a function from the set of terms to (a probabilistic
distribution over) the set of points of the color space. If we let
the game be played repetitively and relate payoffs from each particular
game to the ''fitness'' of the sender and the receiver strategy
employed in that game,%
\footnote{J�ger and van Rooij motivate this by appeal to priming effects between
sufficiently close coding and decoding strategies. We found their
reasoning unconvincing: reinforcement via priming can take place between
the sender and the receiver strategy of a single individual, but not
interindividually, since the sender in a game has no access to the
strategy of the receiver and \textit{vice versa}.%
} or the probability that they will be employed in the next run, we
get an evolutionary process with a specific dynamic. This process
can be, in principle, viewed as a model of evolution of color categories
in a community. How various parameters of such a model are to be set
up is, of course, subject to discussion.

We chose to base our evolutionary model in similarity-maximization
signaling games, rather than in the Category Game of Steels and Belpaeme
\cite{Steels05}, adopted in Baronchelli et al. \cite{Baron10} and
Loreto et al. \cite{Lor12}. In the former setting, categories are
inherently linguistic and can be unproblematically called ''concepts''
as well. The latter approach, on the other hand, makes the conceptual
distinction between perceptual and linguistic categories. Each agent,
based on empirical input, individually divides a continuous perceptual
space into regions (perceptual categories) within which she cannot
further distinguish. A linguistic category then emerges through subsuming
of adjacent perceptual categories under a single term. As perceptual
categorization independent of language seems to be a problematic notion
to us, we prefer the simpler formulation in terms of signaling games.
Admittedly, we consider only the 320 (330?) Munsell chips as the color
space, instead of the continuous space. This choice, motivated by
its simplicity and good evaluability against the WCS data, can be
seen as a preliminary perceptual categorization of the continuous
space. However, a difference is that in our case the perceptual space
is not carved up arbitrarily by individual agents, but uniformly and
in roughly homogeneous way with respect to human resolution abilities.

%\subsection{Empiristic explanation}
%There has been a minor line of research focused at examination of purely empiristic views, trying to  derive realistic categorical systems from the distributions of colors in an individual agents' input. This has not proved fruitful. One can quite straightforwardly perform clustering experiments on color points sampled from realistic living sceneries, which is what has been done. But the ecological significance of a color need not correspond at all to its average share in the agent's visual field, and it is not clear how this factor could be reflected in empiristic experiments. Detached from this, the purely empiristic view has been refuted by Steels and Belpaeme \cite{Steels05},section V, after Yendrikhovskij's \cite{Yendr01} poorly substantiated claim that Berlin and Kay's \cite{Berlin69} color term hierarchy is derivable via empiristic clustering.


%Steels and Belpaeme show that a realistic (as opposed to flat) chromatic distribution helps a population of empiristically learning agents to converge in their categorical repertoires, but is not a sufficiently strong factor to establish a full agreement.\footnote{But cf. Webster and Kay \cite{Webster05}, who in a commentary on Steels and Belpaeme note that full categorical agreement within a population does not occur in the reality of the WCS. To jump ahead, this suggests that the feedback mechanisms which in both game-theoretic settings described in Section ... care for intrapopulational sharing of categories might be too strong.} Even if it was, further, it would not yet present a ready explanation for universal tendencies in color categorization: the milieus of different language communities certainly vary with respect to color distribution (although we might expect certain universal tendencies even here). Moreover, they have found neither a correlation between sets of categories extracted from different realistic sceneries (namely, natural vs. urban), nor a correlation between such sets and the set of �basic human colour categories proposed in the literature�.


%Otherwise, Steels and Belpaeme focus on different and more successful models of arriving at a shared repertoire of categories, particularly on a �culturalistic� model, which involves repetition of standardized linguistic interactions over empirical input.\footnote{They are not specifically interested in the evolutionary universals of color vocabularies.} In this, they only consider flat chromatic distribution and do not examine the potential influence of a realistic distribution on the cultural process. While we think that chromatic distribution as such cannot suffice as an explanation of universal tendencies in color categorization, based on Steels and Belpaeme's results we still believe it to be a possible supportive factor. The model we develop in this paper does not primarily belong to this �empiristic� line of research; however, an extension in this direction suggests itself as soon as we are prepared to accept possible multiple source of the linguistic universals in question.

\subsection{Experimental results}

% latex table generated in R 2.14.1 by xtable 1.7-0 package
% Mon Jan 28 14:07:17 2013
\begin{table}[ht]
\begin{center}
\begin{tabular}{rrrrrrr}
  \hline
 & Min. & 1st Qu. & Median & Mean & 3rd Qu. & Max. \\ 
  \hline
3 & 0.772 & 0.828 & 0.869 & 0.863 & 0.909 & 0.928 \\ 
  4 & 0.328 & 0.569 & 0.658 & 0.644 & 0.713 & 0.834 \\ 
  5 & 0.362 & 0.594 & 0.684 & 0.674 & 0.756 & 0.900 \\ 
  6 & 0.394 & 0.575 & 0.659 & 0.652 & 0.728 & 0.878 \\ 
   \hline
\end{tabular}
\end{center}
\end{table}

% latex table generated in R 2.14.1 by xtable 1.7-0 package
% Mon Jan 28 14:11:12 2013
\begin{table}[ht]
\begin{center}
\begin{tabular}{rrrrrrr}
  \hline
 & Min. & 1st Qu. & Median & Mean & 3rd Qu. & Max. \\ 
  \hline
3 & 0.441 & 0.623 & 0.661 & 0.646 & 0.705 & 0.753 \\ 
4 & 0.306 & 0.455 & 0.503 & 0.513 & 0.578 & 0.681 \\ 
5 & 0.378 & 0.472 & 0.512 & 0.524 & 0.566 & 0.728 \\ 
6 & 0.344 & 0.450 & 0.484 & 0.487 & 0.525 & 0.647 \\ 
   \hline
\end{tabular}
\end{center}
\end{table}

% latex table generated in R 2.14.1 by xtable 1.7-0 package
% Wed Jan 30 22:59:20 2013
\begin{table}[ht]
\begin{center}
\begin{tabular}{rrrrrrr}
  \hline
 & Min. & 1st Qu. & Median & Mean & 3rd Qu. & Max. \\ 
  \hline
3 & 0.375 & 0.542 & 0.588 & 0.582 & 0.631 & 0.744 \\ 
4 & 0.309 & 0.423 & 0.473 & 0.470 & 0.514 & 0.613 \\ 
5 & 0.287 & 0.384 & 0.431 & 0.431 & 0.478 & 0.575 \\ 
6 & 0.319 & 0.416 & 0.450 & 0.450 & 0.484 & 0.625 \\ 
   \hline
\end{tabular}
\end{center}
\end{table}


% latex table generated in R 2.14.1 by xtable 1.7-0 package
% Mon Jan 28 14:11:36 2013
\begin{table}[ht]
\begin{center}
\begin{tabular}{rrrrrrr}
  \hline
 & Min. & 1st Qu. & Median & Mean & 3rd Qu. & Max. \\ 
  \hline
3 & 0.312 & 0.353 & 0.359 & 0.363 & 0.378 & 0.406 \\ 
4 & 0.228 & 0.266 & 0.275 & 0.276 & 0.287 & 0.312 \\ 
5 & 0.175 & 0.225 & 0.237 & 0.236 & 0.250 & 0.272 \\ 
6 & 0.172 & 0.203 & 0.212 & 0.213 & 0.222 & 0.263 \\ 
   \hline
\end{tabular}
\end{center}
\end{table}

% latex table generated in R 2.14.1 by xtable 1.7-0 package
% Mon Jan 28 14:14:40 2013
\begin{table}[ht]
\begin{center}
\begin{tabular}{rrrrrrr}
  \hline
 & Min. & 1st Qu. & Median & Mean & 3rd Qu. & Max. \\ 
  \hline
1 & 0.666 & 0.698 & 0.711 & 0.710 & 0.724 & 0.750 \\ 
  2 & 0.534 & 0.552 & 0.594 & 0.587 & 0.614 & 0.641 \\ 
  3 & 0.506 & 0.554 & 0.641 & 0.605 & 0.643 & 0.675 \\ 
  4 & 0.641 & 0.649 & 0.664 & 0.662 & 0.677 & 0.681 \\ 
  5 & 0.441 & 0.447 & 0.463 & 0.477 & 0.502 & 0.541 \\ 
  6 & 0.656 & 0.697 & 0.698 & 0.702 & 0.716 & 0.741 \\ 
  7 & 0.622 & 0.645 & 0.658 & 0.665 & 0.680 & 0.725 \\ 
  8 & 0.659 & 0.691 & 0.705 & 0.704 & 0.718 & 0.744 \\ 
  9 & 0.581 & 0.627 & 0.634 & 0.642 & 0.654 & 0.719 \\ 
  10 & 0.662 & 0.693 & 0.711 & 0.708 & 0.722 & 0.753 \\ 
   \hline
\end{tabular}
\end{center}
\end{table}


% latex table generated in R 2.14.1 by xtable 1.7-0 package
% Mon Jan 28 14:20:02 2013
\begin{table}[ht]
\begin{center}
\begin{tabular}{rrrrrrr}
  \hline
 & B�t� & Ejagam & Kwerba & Nafaanra & Wob� & Yacouba \\ 
  \hline
1 & 0.728 & 0.713 & 0.750 & 0.666 & 0.694 & 0.709 \\ 
  2 & 0.619 & 0.588 & 0.641 & 0.541 & 0.534 & 0.600 \\ 
  3 & 0.644 & 0.641 & 0.675 & 0.506 & 0.525 & 0.641 \\ 
  4 & 0.641 & 0.672 & 0.647 & 0.656 & 0.678 & 0.681 \\ 
  5 & 0.541 & 0.512 & 0.441 & 0.469 & 0.456 & 0.444 \\ 
  6 & 0.722 & 0.700 & 0.741 & 0.656 & 0.697 & 0.697 \\ 
  7 & 0.684 & 0.666 & 0.622 & 0.650 & 0.725 & 0.644 \\ 
  8 & 0.722 & 0.706 & 0.744 & 0.659 & 0.688 & 0.703 \\ 
  9 & 0.637 & 0.659 & 0.581 & 0.631 & 0.719 & 0.625 \\ 
  10 & 0.725 & 0.713 & 0.753 & 0.662 & 0.688 & 0.709 \\ 
   \hline
\end{tabular}
\end{center}
\end{table}



\bibliographystyle{plain}
\bibliography{bibl}
 
\end{document}
